%% utokyo eeis rinko template created by Ryoma Hirata

% documentclass
\documentclass[a4paper,10.5pt,dvipdfmx]{bxjsarticle}

\usepackage[dvipdfmx]{graphicx} % 画像の取扱いに必要
\usepackage[dvipdfmx]{hyperref} % ハイパーリンクに必要
\usepackage{fancyhdr} % ヘッダー・フッターの設定に必要
\usepackage{lipsum} % ダミーテキスト生成に必要
\usepackage{bxjalipsum}

\usepackage{cite}
\usepackage{multicol}
\usepackage{geometry}
\usepackage{caption}
\usepackage{amsmath}
\usepackage{amssymb}
% \usepackage{wrapfig}

\geometry{top=10mm,bottom=10mm,left=15mm,right=15mm}

\pagestyle{fancy}
\fancyhf{} % ヘッダー・フッターの初期化

% フッターの設定
\cfoot{\thepage} % ページ番号を中央に表示

\renewcommand{\headrulewidth}{0pt}

\begin{document}

\thispagestyle{fancy}

\noindent
\fbox{
  \begin{minipage}{\dimexpr\linewidth-2\fboxsep-2\fboxrule\relax}
    電気系修士輪講資料
    \hfill
    2025年 11月 14日
    % タイトル
    \begin{center}
      {\LARGE \bfseries スパイキングニューラルネットワークを用いた\\ドローン飛行制御に関する研究調査 \par}
      \vspace{2mm}
      {\large A Survey on Spiking Neural Network for Drone Flight Control \par}
    \end{center}

    \vspace{5mm}

    \noindent
    指導教員: 福田盛介 教授
    \hfill
    修士課程1年 37-256564 平田涼馬
  \end{minipage}
}

\vspace{5mm}

% アブストラクト
\begin{center}
  \bfseries Abstract
\end{center}
AI-based flight control for drones has gained significant attention for applications in unknown and dynamic environments. 
Particularly, in environments where communication delays are significant, such as Mars, there is a need for AI models that
can operate onboard, and the application of Spiking Neural Networks (SNNs) is anticipated. This survey analyzes recent research
related to SNN-based drone flight control, focusing on (1) methods using reinforcement learning, (2) methods using imitation learning,
and (3) methods combined with event-based vision sensors. Finally, we summarize and compare the key results of each approach and
discuss their future prospects to the flight control of Mars exploration drones.

\begin{multicols}{2}

% 本文
\section{序論}
ドローンは地形や障害物の影響を受けにくい高い移動能力や導入・運用コストの低さなどの特徴から農業や建設,
物流,惑星探査など幅広い用途への応用が進んでいる.これらの応用において,高い自律性を持つドローンの開発が
求められている.ドローンの飛行制御には,従来の制御理論に基づくPID制御やモデル予測制御などが用いられる
ことが一般的であったが,これらの手法には未知の環境や動的な環境への適応性,複雑なタスクの遂行能力に限界
がある.これらの課題に対処するため,近年ではニューラルネットワークを用いた制御手法に注目が集まっている
\cite{mcenroe2022survey}.ニューラルネットワークを用いたドローン制御ネットワークの推論は,通信を用いて
高性能なコンピューターにデータを送信して推論を行う方法と,ドローンに搭載したエッジデバイスで行う方法に大別
される.通信を用いる場合,高い計算能力を利用して高度な行動計画を立てたり,高い精度での物体認識などを活用
した制御が可能である一方で,通信遅延の影響からリアルタイム性に欠けるという課題がある.一方,エッジデバイス
を用いる場合,リアルタイムでの制御が可能であるが,電力の使用が飛行時間の短縮につながってしまう点や,
計算能力の不足からモデルの軽量化を行った結果,推論精度が低下するという課題がある\cite{rezwan2022artificial}.
これらの課題に対処するため,省電力かつ高効率な計算が可能なスパイキングニューラルネットワーク(SNN)を
用いたドローンの飛行制御手法が注目されている.

% \begin{center}
%   \includegraphics[width=\linewidth]{figures/SNN_operation.png}
%   \captionof{figure}{スパイキングニューラルネットワークの概念図 (\cite{SnnfromScratch}を一部改変)}
%   \label{fig:snn_operation}
% \end{center}

\subsection{スパイキングニューラルネットワーク}
スパイキングニューラルネットワーク(SNN)とは,生物の脳が{0,1}のスパイクを用いて情報伝達することを模倣した
AIモデルである.このスパイクを処理するニューロンのモデルとして使用されているのがLIFモデルである.中でも
Current based-LIFモデル(Cuba-LIF)は,シナプス電流と膜電位の2つの状態の時間変化を考慮したモデルであり,
以下の式で表される.

\begin{equation}
  \begin{split}
    v_i(t+1) = \tau^{mem}_i v_i(t) + i_i(t), \\
    i_i(t+1) = \tau^{syn}_i i_i(t) + \sum_j W_{ij} s_j(t)
  \end{split}
  \label{equ:cuba_lif}
\end{equation}

ここで,$v(t)$は膜電位,$i_i(t)$は入力電流,$\tau_{mem}$は膜時定数,$\tau_{syn}$はシナプス時定数を表す.
スパイクが入力されると,膜電位が上昇し,入力が無い場合は静止膜電位に向かって減衰する.連続でスパイクが
入力されることで,膜電位が閾値を超えるとニューロンが発火し,出力スパイクを生成する.このCuba-LIFモデル
は,IntelのニューロモルフィックチップであるLoihiのデフォルトのLIFモデルであり,広く使われている\cite{davies2018loihi}.
SNNの特徴として,入出力情報が{0,1}のスパイクで表現されることや,スパイクの入出力がある場合のみニューロンが
活動するため,専用のニューロモルフィックチップ上に実装した場合,計算量が少なく,省電力,リアルタイム処理が
可能であることが挙げられる.一方で,SNNは入出力で扱うスパイクが微分不可能であるため,ANNで一般的に
用いられる誤差逆伝搬法を直接適用できないため,学習が困難であるという課題がある.現状,SNNの学習手法
としては,ANNで学習したパラメータをSNNに変換する手法や,代理勾配を用いて誤差逆伝搬法を適用する手法などが
用いられているが,ANNと比較して学習効率が低いという課題がある\cite{yamazaki2022spiking}.また,
ニューロモルフィックチップの性能も発展途上であり,ANNと比較してチップの計算能力が劣る場合が多い.

\subsection{火星探査への展望}
火星探査において,ドローンは地形の把握や科学観測,通信中継など多様な役割を担うことが期待されている.
NASAの火星ドローン「Ingenuity」は,火星大気中での飛行実証を成功させ,その後も複数の飛行ミッションを
遂行している\cite{Ingenuity}.将来的な火星探査ミッションでは,ドローンがより高度な自律飛行能力を
持つことが求められており,未知の地形や動的な環境に適応できる制御手法の開発が必要である.この制御手法
として,SNNを用いたドローン制御は,省電力かつリアルタイム処理が可能であるため,火星探査ドローンへの
応用が期待されている\cite{harbour2024martian}.

\begin{center}
  \includegraphics[width=\linewidth]{figures/Ingenuity_width.png}
  \captionof{figure}{Ingenuity Mars Helicopter \cite{Ingenuity}}
  \label{fig:ingenuity_on_mars}
\end{center}

\subsection{本調査の目的}
本調査では,ドローンの飛行制御において,SNNのネットワークを姿勢安定化や移動などの低レベル制御に用いた
2つの研究と,動的な環境での経路計画を含めたナビゲーションにSNNを用いた1つの研究について紹介する.1本目の研究では,
IMUセンサーを用いて姿勢推定と制御を行うことを提案したNeuromorphic attitude estimation and control
\cite{stroobants2025neuromorphic}を紹介する.2本目の研究では,入力にイベントカメラを使用し,自己
教師あり学習と進化アルゴリズムを組み合わせた複合学習により視覚ベースの自己運動制御を行うことを提案した
Fully neuromorphic vision and control for autonomous drone flight\cite{paredes2024fully}を
紹介する.3本目の研究では,強化学習アルゴリズムであるPPOによりSNNを学習させてドローンの高機動ナビゲーション
を行うことを提案したBio-Inspired Drone Control A Reinforcement Learning-Trained Spiking Neural Networks for Agile Navigation in Dynamic Environment
\cite{lee2025bio}を紹介する.最後に,各手法の分析から将来の火星探査ミッションにおけるドローン飛行制御
への応用可能性について議論を行う.

\section{SNNによるドローンの低レベル制御}
\subsection{IMUと模倣学習による姿勢推定・制御}
\subsubsection{概要}
この研究では,ドローンの自律制御を単一のニューロモルフィックチップのみで実現することを目的として,
IMUの入力からモーターコマンドを出力するSNNベースの制御システムを提案している.本手法では,学習時に
ネットワークを姿勢推定と制御の2つのサブネットワークに分割し,それぞれを教師あり学習,模倣学習で訓練
すること,SNNを用いることにより生じるセンサーバイアスやフィードフォワード遅延に対象するための工夫を
行っている.また,提案手法はCrazyflie上に実装され,評価が行われた.

\subsubsection{ネットワーク構成}
本手法では、ネットワークへの入力としてIMUを用いている.IMUは加速度センサーとジャイロスコープから
構成され,ドローンの3自由度の加速度と角速度を提供する.ネットワーク全体の構成は図\ref{fig:neuro_network}
に示す通りである. \\
2つに分割されたサブネットワークの前段では,IMUの6つの入力から,ドローンの姿勢の
推定値が算出された.次に,ネットワークの後段では姿勢の推定値と目標の姿勢が入力され,モーターの
トルクコマンドが出力された.この
\begin{center}
  \includegraphics[width=\linewidth]{figures/neuro_network.png}
  \captionof{figure}{SNN network architecture \cite{stroobants2025neuromorphic}}
  \label{fig:neuro_network}
\end{center}
ネットワークに用いられたニューロンモデルはシナプス電流の時間変化を
考慮したCUBA-LIFモデルであり,学習には代理勾配を用いたBack Propagation Through Time(BPTT)が用いられた.
ここで扱われる入出力はスパイクであるが,ネットワークの入出力は連続値であるため,線形層を用いた変換
が行われている.これにより,スパイクのエンコードとデコードを学習に組み込むことで,最適化された変換
を可能にしている.また,学習時には分割されている,姿勢推定ネットワークの出力の線形層の重み$W_o$と,
制御ネットワークの入力の線形層の重み$W_i$は,推論時には以下の式で表されるように統合され,一つの
ネットワークとして動作する.

\begin{equation}
  \begin{bmatrix}
    \phi_{\text{est}} \\ \theta_{\text{est}} \\ \psi_{\text{est}}
  \end{bmatrix}
  = W_{\text{i}} W_{\text{o}} s(t)
  \label{equ:mearge_networks}
\end{equation}

\subsubsection{SNNの学習方法}
SNNの学習には模倣学習が用いられた.学習データは,姿勢推定に相補フィルタ,制御にカスケードPID制御を使用し
手動で20分間飛行させ,500Hzで収集された.学習には,教師ありBPTTが用いられ,損失関数は以下の式で表される
平均二乗誤差(MSE)とピアソンの相関損失を組み合わせたものとして定義された.

\begin{equation}
  J(p) = \text{MSE}(x, \hat{x}) + \frac{1}{2}(1 - \rho(x, \hat{x}))
  \label{equ:neuro_loss_function}
\end{equation}

また,非連続なスパイク関数の微分を近似するために,代理勾配が用いられた.本手法では,スケール付き
アークタンジェントの導関数であり,以下の式で表される.

\begin{equation}
  \frac{d}{dx} \left(\frac{1}{s} \arctan(s x) \right) = \frac{1}{1 + (s x)^2}
  \label{equ:arctan_surrogate}
\end{equation}

SNNは,電流と電圧のリークを考慮するため,出力が出されるまでにニューロンに一定以上の入力が蓄積される必要
があり,出力に遅延が生じる.
この遅延を補償するために,学習データのラベルを6ステップ先の値にシフトさせて学習が行われた.
さらに,模倣学習を用いた事による現実とのギャップを低減するために,学習データの拡張が行われた.具体的には,
(ⅰ)初期の学習データで訓練されたSNNを用いて飛行させ,その時に学習データ用のコントローラーで得られるはず
だったデータ,(ⅱ)通常のPIDのコントローラーにランダムな外乱を加えたデータが収集され,学習
データに追加された.これらのデータ拡張により,より広範な状態空間での学習が可能となり,現実世界での適応性
を向上させている.
また,制御ネットワークの学習において,定常的な誤差を低減するためのI制御の学習は,(ⅰ)ニューロンの漏れに
より長期間で情報を蓄積できないこと,(ⅱ)PD制御の成分の学習が優先されることから困難である.これに対して,
制御ネットワークの一部のニューロンのパラメーターを固定し,リークをなくすことで膜電位が積分として機能する
ようにし,そのニューロンの出力の正解ラベルをI制御の出力のみに設定して学習を行った.

\subsubsection{制御精度の評価}
位置移動の能力を評価するために,提案手法を用いて1m前進し原点に戻るタスクを実施し,通常のPID制御との比較が
行われた.タスクは10回施行され,提案手法は全ての施行でPID制御と同等の性能を示し,安定した目標追従が可能な
ことが示された.
次に,各学習方法によるSNN制御とPID制御制御による姿勢応答の比較を行うため,ロールを$0°$,$+10°$,$-10°$,$0°$と
変えて行ったときの応答の比較が行われた(図\ref{fig:neuro_comp}).提案手法において,ロール角の目標値と
制御値のRMSEは3.03°であり,姿勢推定及び制御が正確に行われてことが示された.また,PIDのRMSEは2.67°であり,
提案手法より高い精度であったが,これは外部から姿勢情報を直接与えられており,制御のみが誤差の要因であるため
だと考えられる.提案手法では,姿勢推定と制御の両方が誤差の要因となるため,このことからも提案手法は十分な
精度を達成していることが示された.

\begin{center}
  \includegraphics[width=\linewidth]{figures/neuro_comp.png}
  \captionof{figure}{Attitude step responses of A) the fully-trained SNN system, B) the SNN trained with augmentation, 
  C) the SNN trained with time-shifted data and D) the regular PID flight stack. \cite{stroobants2025neuromorphic}}
  \label{fig:neuro_comp}
\end{center}

\subsubsection{消費電力の分析}
最後に,本手法における消費電力の分析が行われた.提案手法は使用した小型クアッドローターに対して,現在利用
可能なニューロモルフィックチップが大きすぎるため,従来のマルチプロセッサにて実装が行われた.一方で,SNN
を用いることによる省電力性は単一のニューロモルフィックチップで動作させた場合に得られる.そのため,潜在的
な消費電力の削減効果を評価するために,ネットワークが行う演算数に基づいて,消費電力の推定が行われた.その
結果,PIDベースの制御は3000回の加算,提案手法は7500回の加算が必要と等価と見積もられ,SNN制御器はPID制御
と同じ桁のエネルギー効率を達成することが推定された.このエネルギー効率は,将来的にシステムが画像処理タスク
と統合された場合に,スパイクの疎性により乗算を削減できるため,向上すると考えられている.また,イベントベース
制御への拡張により,ホバリングなどの制御を行わない場合に,更に消費電力を削減できる可能性があると示唆された.

\subsubsection{考察と今後の課題}
本手法で示された,SNNの学習時のデータ拡張や,積分動作のニューロンによる実装などは,手法を拡張させる際にも
有効であると考えられる.一方で,ラベルをシフトさせる手法は,ドローンがより複雑な軌道で飛行する場合や,移動速度が
早くなる場合には適切に補償できなくなる可能性がある.そのため,より一般的に適用可能な遅延補償手法の検討が
必要であると考えられる.また,示された電力効率はPID制御と同等の桁であることが主張されたが,2倍以上の計算量であり,特にバッテリー
量が限られる火星ドローンにおいては無視できない差である.そのため,画像を用いた制御と統合した場合の消費
電力と精度のトレードオフの分析が必要であると考えられる.

\subsection{イベントベース視覚の複合学習による自己運動制御}
\subsubsection{概要}
この研究は,イベントカメラを用いて取得したイベントベースの視覚情報を入力とし,低レベルの制御アクションを
出力するSNNを提案している.ネットワークは視覚処理部と制御部にモジュール分割されており,視覚処理部は自己
教師あり学習,制御部はシミュレーター上で進化アルゴリズムを用いた学習が行われた.また,提案手法はIntel 
Loihiプロセッサを用いて実装され,評価が行われた.

\subsubsection{イベントカメラの特性}
本手法では,視覚処理を行うためのセンサーとしてイベントカメラを用いている.イベントカメラは,従来のフレーム
ベースのカメラとは異なり,ピクセル毎に独立して輝度変化を検出し,変化があった場合のみイベントとして出力する
センサーである.このカメラの特徴として,(ⅰ)高い時間分解能,(ⅱ)低消費電力,(ⅲ)高いダイナミックレンジが
挙げられる\cite{gallego2020event}.

\subsubsection{イベントベースのオプティカルフロー}
オプティカルフローとは,連続する画像間での物体やテクスチャがどう動いたかをベクトルで表したものである.
イベントカメラからの入力に対してSNNを用いてオプティカルフローを推定する手法も提案されており,
エンコーダー・デコーダーモデルを用いた自己教師あり学習が用いられている\cite{hagenaars2021self}.
ここで,自己教師あり学習が用いられるのは,オプティカルフローの正解ラベルをピクセル単位で大量に集めること
が困難であるためである.SNNを用いた手法は,ANNと同等の水準を達成したことが示されているが,現状入手可能な
ニューロモルフィックチップ上に実装するにはネットワークのニューロン数が多すぎるという課題があり,実際の
視覚処理に組み込もうと考える場合は,精度を維持しながらネットワークのサイズを削減する工夫が必要となる.

\subsubsection{提案システムの概要}
この研究では,図\ref{fig:fully_system}に示すシステムが提案された.ドローンには,イベントカメラ,ニューロ
モルフィックプロセッサ,シングルボードコンピュータおよびフライトコントローラーが搭載され,イベントカメラ
からの情報を制御コマンドに変換するSNNがニューロモルフィックプロセッサ上で実行される.SNNの視覚処理部は,
Loihiチップ上でイベントベースのオプティカルフローを実装する上で課題となる計算リソースの制約を克服するために,
イベントカメラは静的でテクスチャが豊富な平坦な表面を見下ろしていることが前提となっている.この条件で,
イベントカメラの画像平面の内,コーナーの関心領域からの情報のみが用いられ,更にROIあたりのイベント数を90
に制限することで,200Hzの動作周波数が実現されている.

\begin{center}
  \includegraphics[width=\linewidth]{figures/fully_system.jpg}
  \captionof{figure}{Overview of the proposed system (\cite{paredes2024fully})}
  \label{fig:fully_system}  
\end{center}

\subsubsection{SNNの学習手法}
本手法のSNNに用いられたニューロンモデルはCUBA-LIFモデルであり,学習には代理勾配が用いられた.
視覚処理を行うネットワークは3つのエンコーダーとプーリング層で構成され,4つのROIからオプティカル
フローが推定される.このフローから密なオプティカルフローを得るために以下の式が用いられた.

\begin{equation}
  \mathbf{u}(\mathbf{x},\mathbf{H}) = \begin{bmatrix} u(\mathbf{x}, \mathbf{H}) \\ v(\mathbf{x}, \mathbf{H}) \end{bmatrix} 
  = \left(\mathbf{H}, \begin{bmatrix} x \\ y \\ 1 \end{bmatrix} \right)_{\text{Eucl}} - \begin{bmatrix} x \\ y \end{bmatrix}
  \label{equ:dense_optical_flow}
\end{equation}

ここで,$\mathbf{H}$はホモグラフィ行列,$\mathbf{x} = (x, y)$は画像平面上のピクセル座標を表す.
また,このホモグラフィ行列は,ある時間間隔におけるROI内のイベントの変位から計算される.
この密なオプティカルフローを元に,動き補償のためのコントラスト最大化が行われた.
以下にはネットワークの学習に用いられた損失関数を示した.

\begin{equation}
  L_{\text{flow}} = L_{\text{contrast}} + \lambda L_{\text{smooth}}
  \label{equ:flow_loss_function}
\end{equation}

この式(\ref{equ:flow_loss_function})の内,$L_{\text{contrast}}$は予測した動きを元に
ブレを補正した後のフロー画像がどれだけ正確であったかを評価するための損失である.また,
$L_{\text{smooth}}$はフローのスムーズさを評価するための正則化項であり,$\lambda$はその重みを表す.

次に,制御は視覚処理での推定値を線形変換することで行われた.この推定値とは,3軸の速度と角速度,
ピッチとロール,目標速度の9次元のベクトルであり,推力と目標とするロール,ピッチ,ヨーの4次元の
制御コマンドを出力するための$4\times9$の行列の重みが学習された.この学習には突然変異のみの進化
アルゴリズムが用いられた.具体的には,100個のランダムな$4\times9$行列の集団を用意し,
$\mathcal{N}(0,0.001)$のノイズを加えて適応度の評価を行い,適応度が上位100個の行列を次世代と
して,25,000世代まで学習が行われた.この際の適応度は以下の式で定義された.

\begin{equation}
  \scriptsize
  F = \frac{1}{N_{\text{eval}}} \sum_{i \in N_{\text{eval}}} \sum_{j \in N_{\text{steps}}} 
  \mathbf{w} \cdot \left( v^{\mathcal{B}}_{\text{sp},i} - \begin{bmatrix} \hat{V}^
    {\mathcal{B}}_{x} \\ \hat{V}^{\mathcal{B}}_{y} \\ \hat{V}^{\mathcal{B}}_{z} \end{bmatrix}_{j}
     \right)^{2} + \left( \hat{\omega}^{\mathcal{B}}_{z} \right)^{2}
  \label{equ:fitness_function}
\end{equation}

ここで,$N_{\text{eval}}$は評価回数,$N_{\text{steps}}$は各評価におけるステップ数,$\mathbf{w}$は
重みベクトルであり,z垂直方向の誤差を10倍重視するように設定されている.

\subsubsection{シミュレーションと実飛行での制御精度の比較}
提案手法の評価は,視覚部と制御部で分けて行われた.まず,視覚部の評価では屋内環境でドローンの下部に
搭載されたイベントカメラを用いて撮影されたイベントシーケンスを用いて学習を行い,同時に計測した
ドローンの正確な位置と姿勢情報との比較が行われた.

\begin{center}
  \includegraphics[width=\linewidth]{figures/fully_attitude.png}
  \captionof{figure}{Comparison of estimation and ground-truth visual observables for sequences with different motion speeds \cite{paredes2024fully}}
  \label{fig:fully_attitude}
\end{center}

図\ref{fig:fully_attitude}は,低速,中速,高速の3つの速度でのSNNによる速度の推定結果と正解ラベルの
比較を示しており,どの速度でもネットワークは動きを正確に捉えることが出来ていることが示されている.
特に,回転速度の推定においては,毎秒約4ラジアンの高速回転も推定可能であることが示されている.また,
推定値が含む誤差に関しては,1つのカメラから運動の推定を行うことによるアパーチャ問題が主な原因
となっていると考えられている.また,推定したフローの平均エンドポイント誤差を元にした分析も行われた.
まず,ANNの手法(Conv-GRU)とSNNを比較した場合,ANNが5.88,SNNが10.79となり,ANNの方が精度が高い
ことが示された.これは,floatを用いるANNのほうが高精度な計算が可能であるためだと考えられる.次に,
画像全体を用いた場合と,ROIのみを用いた場合の比較において,SNNではROIのみの場合が精度が向上した
ことが示された.イベント数の制限でも同様に精度の向上が見られた.これは,ネットワーク内に膜電位として
蓄積される情報が増えすぎる事による処理の不安定化が解消されたためだと考えられる.

次に,制御部の評価はシミュレーションと実飛行の両方で行われた(図\ref{fig:fully_control}).
シミュレーションでは,16パターンの飛行目標が与えられ,学習が行われた.その結果,16パターン全てに
おいて目標への到達が可能なことが示された.次に,実飛行での評価では,飛行には成功したものの多くの
パターンで目標の姿勢,速度への到達を達成出来なかった.これは,単純な線形コントローラーを用いたこと
による表現力の不足や,リアリティギャップ,積分制御の補正不足が原因であると考えられている.また,
高速で飛ぶ場合の方が飛行軌跡が安定することも示された.これは,イベントカメラが生成するイベントが
高速な場合ほど多くなるため,ノイズの影響が低減されたことが理由だと考えられている.また,PI制御器
との比較においても上記の課題から,PI制御器の方が優れた性能を示した.

\begin{center}
  \includegraphics[width=\linewidth]{figures/fully_control.jpg}
  \captionof{figure}{Comparison of results obtained in simulation and during real-world flight tests \cite{paredes2024fully}}
  \label{fig:fully_control}
\end{center}


\subsubsection{推論時間と消費電力}
推論時の速度と消費電力の評価を行うために,提案手法をLoihiチップとNVIDIA Jetson Nano上で動作させた
場合の比較が行われた.LoihiチップでのSNN実行時の消費電力は0.95Wであるのに対し,Jetson Nanoの10Wモード
で動作させた場合の消費電力は2.98Wであった.さらに,Loihiでの消費電力の内,0.9Wはアイドル時の電力で
あり,これを削減することでより低消費電力化が可能であると示唆された.また,推論速度も1秒間の推論回数
において,ドローンが低速な場合,Loihiでの推論は1637inf/sであるのに対し,Jetson NanoではANNを動作
させた場合においても,60inf/sであり,Loihiの方が大幅に高速であることが示された.

\subsubsection{考察と今後の課題}
本手法では,ドローンの視覚制御をオンチップで200Hzで実現することが出来ており,SNNを用いた視覚制御
の有効性が示された.特に,イベントデータからのフロー推定について,軽量化のために行った,ROIによる
情報の制限が精度の向上に繋がったことは有用な結果であると考えられる.一方で,本手法で想定された
安定した環境下での飛行は,実応用においては重大なリアリティギャップを生じる可能性があり,ドローンの
回転方向の動きが複雑化した場合や,床面のテクスチャが乏しい場合などに現状の視覚処理モデルでどの程度
の性能が維持できるかは不明である.そのため,より多様な環境下での評価や,視覚処理モデルの改良が必要
であると考えられる.

\section{SNNによる高機動ナビゲーション}
\subsection{概要}
本研究では,深層強化学習アルゴリズムである近位ポリシー最適化(PPO)を使用してSNN飛行ポリシーの学習が
行われた.この飛行ポリシーでは,システムの状態をドローンの低レベルの制御コマンドに変換することを提案
している.本手法はシミュレーション上でリングを回避しながら進むタスクにて,SNNの低計算量で時間情報を処理
することが可能な特性を活かし,ANNと比べて高い成功率で,完了時間も短縮可能なことが示された.

\subsection{PPOアルゴリズムの概要}
近接ポリシー最適化(PPO)は,強化学習アルゴリズムの一つであり,信頼性と効率性を両立させながら実装が
シンプルであることが特徴である\cite{schulman2017proximal}.PPOでは,目的関数として以下の式で表される
クリップ付き代理目的関数が用いられる.

\begin{equation}
  \scriptsize
  \displaystyle
  L^{CLIP}(\theta) = \hat{\mathbb{E}}_t \left[ \min \left( r_t(\theta) \hat{A}_t, \text{clip}(r_t(\theta), 1 - \epsilon, 1 + \epsilon) \hat{A}_t \right) \right]
  \label{equ:ppo_objective}
\end{equation}

この式では,ポリシーの更新において,古いポリシーと新しいポリシーの比率$r_t(\theta)$が1から大きく
外れないように制約を設けることで,安定した学習を実現しており,様々なタスクで高い性能を発揮している.

\subsection{PPOアルゴリズムによるSNNエージェントの学習}
提案されたフレームワークでは,物理シミュレーション環境,PPOアルゴリズムで学習されたSNNベースのエージェント,
低レベルコントローラーが組み合わされている(図\ref{fig:bio_training}).高速で移動するゲートを通過するタスクにおいて,SNNエージェントは
シミュレーションステップ毎に,システムの状態$s \in \mathbb{R}^{19} := (d_{tgt}, q, v, \omega, d_{gate}, v_{gate})$
を受け取る.ここで,$d_{tgt}$は目標位置までの距離,$q, v, \omega$はドローンの姿勢,速度,角速度,$d_{gate}$
はドローンと次のゲートの中心との距離,$v_{gate}$はゲートの中心の速度を表す.このシステムの状態を元に,SNN
エージェントは低レベルなフライトコントローラーのコマンドとなる共通推力とボディーレートを出力する.このとき,
報酬関数によって定義される報酬信号$r$が与えられる.SNNは,1024個のニューロン持つ2つの隠れ層から構成されており,
シグモイド代理勾配関数を用いて学習が行われた.次に,報酬関数には以下の式が用いられた.

\begin{equation}
  r_t = r_{gate,t} + r_{pos,t} + r_{pos,t}\cdot(r_{up,t} + r_{spin,t}) - \beta \cdot r_{acc,t}
  \label{equ:reward_function}
\end{equation}

ここで,$r_{gate,t}$はゲート通過に対する報酬,$r_{pos,t}$はドローンの位置の報酬であり,
以下の式で定義される.

\begin{equation}
  r_{pos,t} = \frac{0.05}{1 + d^{2}_{tgt}} + \frac{2.5}{1 + d^{2}_{gate}} - \frac{2.0}{1 + d^{2}_{ring}}
  \label{equ:position_reward}
\end{equation}

この式は,ドローンの位置の報酬は,最終的な目標地点と通過するゲートの中心に近いほど高くなり,リングに近いほど
低くなることを示している.また,$d_{tgt}$と$d_{gate}$の係数の比率から,最終地点への到達が早いほど報酬が最大化
されるように設計されている.さらに,$r_{up,t}$と$r_{spin,t}$はドローンの姿勢を水平に保つための報酬,$r_{acc,t}$
は急激な加速度を抑制するための罰則項であり,ドローンの軌道を滑らかにするために導入されている.シミュレーション
環境では,0.5~9.0mのランダムな距離で配置された5つのゲートを通過するタスクが設定された.また,各ゲートは,Y-Z平面
内で振動するように設定された.

\begin{center}
  \includegraphics[width=\linewidth]{figures/bio_training.png}
  \captionof{figure}{Schematic structure of the training flow \cite{lee2025bio}}
  \label{fig:bio_training}
\end{center}

\subsection{SNNとANNの比較}
提案手法との比較を行うベースラインとして,ELU活性化関数を備えた4層からなるANNアーキテクチャが用いられた.まず,
タスクの成功率の比較において,提案手法はすべてのゲート間距離の設定において,ANNベース手法よりも高い平均報酬を
達成した.これは,(ⅰ)高い成功率,(ⅱ)各エピソードの完了時間の短縮に起因するものである.まず,成功率の比較に
ついて,SNNはSOTAであるANNの平均成功率である$83\%$を上回る$85\%$を達成した(図\ref{fig:bio_performance}(b)).
特に,ゲート間距離が$0.5m$の場合と初期速度が高い場合において,ANNは成功率が大きく低下したが,SNNは安定した
成功率を維持した(図\ref{fig:bio_performance}(c)).これは,SNNの方が急速な状態変化への適応性に優れていることを
示している.この理由として,SNNのニューロンが膜電位として時間情報を保持できるため,過去の状態に基づいて現在の
行動を調整できることが挙げられる.次に,完了時間の比較において,ゲート間隔が長くなるに連れて,SNNはANNよりも
短い完了時間であることが示された(図\ref{fig:bio_performance}(a)).これについて,SNNはANNよりも軌道全体で
高い速度が維持されていたことが示された.これは,SNNが時間情報を適切に処理できることにより,ゲート通過時の
加減速が効果的に行われたためであると考えられる.

\begin{center}
  \includegraphics[width=\linewidth]{figures/bio_performance.png}
  \captionof{figure}{Comparison between SNN (our method) and ANN (SOTA) across three performance 
  metrics \cite{lee2025bio}}
  \label{fig:bio_performance}
\end{center}

\subsection{エネルギーとメモリの分析}
提案SNNモデルは,1024個のニューロンを持つ隠れ層が2つで構成されており,ニューロンの総数は2048個であった.
この内,$15~20\%$が活性化されており,ニューロンが23個の入力層と4個の出力層を含めた場合の実際の演算数
は,約215,000回であると見積もられる.また,モデルのメモリ使用量は,シナプス接続毎の重み,層ごとの時定数,
ニューロン毎のバイアスをモデルパラメータとし,float32で表現した場合,約4.3MBであると見積もられた.また,
これらのパラメーターを量子化し,int8で表現した場合,約1.08MBに削減できると見積もられた.

\subsection{考察と今後の課題}
本手法の結果から,提案されたSNNモデルはSNNの時間情報を処理する能力から,動的な環境におけるナビゲーションに
優れた適応性を持つことが示された.一方で,火星環境への応用を考えた場合に,突風などの外乱を除けば,静的な
環境であるため,動的な環境への対応よりも,未知の環境への適応性が重要になると考えられる.そのため,用いる
報酬関数もそれに応じた設計へと変更することが必要である.次に,本手法で行われた実験はシミュレーション上の
ものであり,入力はシステムの状態という理想的なものであった.そのため,実環境におけるドローンの飛行制御に
適用しようとする場合,システムの状態を推定するための方法が必要となる.SNNの性質を最大限に活用するためには,
イベントベースの視覚処理と組み合わせたニューロモルフィックな構成を作ることが望ましいが,視覚処理をSNNに
含める場合,計算リソースの制約を満たすことができるのかについては検討が必要である.

\section{まとめ}
本調査では,ドローンの飛行制御において,適応性と堅牢性を省エネルギーで向上させるためのアプローチとして,
SNNを用いた3つの論文を紹介した.
まず,ドローンの姿勢推定について2本目の研究から,イベントカメラを用いた構成を用いる場合,低消費電力で
視覚処理をリアルタイムに行えることが示されたが,入手可能なニューロモルフィックチップの計算能力の制約
から,視覚情報の一部のみを用いるなどの工夫が必要であることが示された.この工夫による姿勢推定の精度の
悪化については議論されていないが,飛行実験では制限された環境でのデータ収集が行われていたため,火星環境
への応用を考える場合,姿勢推定に十分な情報が得られるかは不明である.そのため,一本目の研究で紹介された
IMUと組み合わせたマルチモーダルなアプローチなどによる精度の向上が検討されるべきであると考えられる.
次に,制御へのSNNの応用に関しては,各学習方法における特徴が反映されると考えられる.模倣学習は,特に
リアリティギャップの低減が可能であることが特徴であるが,学習データの多様性への依存性からデータの拡張
が重要であると考えられる.一方,進化アルゴリズムは今回の適用例では単純な線形層の学習への適用に留まって
おり,モデルサイズをより小さくすることが求められる場合には,最適かもしれないが,複雑な環境への適応性
を求める場合の制御コントローラーの学習には十分な精度が得られないという懸念がある.最後に強化学習は,
未知の環境への適応性が高いことが特徴であるが,この特徴を最大限に活用するためには,適切な報酬関数の設計
が重要である.今後の研究では,火星表面の風による外乱への堅牢性を獲得するために,外乱を含む
シミュレーション環境での強化学習の実施や,イベントカメラによる視覚処理と強化学習を組み合わせた制御モデル
を構築することを目指したいと考えている.

\small
\bibliographystyle{my_junsrt}
\bibliography{references}

\end{multicols}

\end{document}
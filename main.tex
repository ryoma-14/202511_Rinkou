%% utokyo eeis rinko template created by Ryoma Hirata

% documentclass
\documentclass[a4paper,10.5pt,dvipdfmx]{bxjsarticle}

\usepackage[dvipdfmx]{graphicx} % 画像の取扱いに必要
\usepackage{fancyhdr} % ヘッダー・フッターの設定に必要
\usepackage{lipsum} % ダミーテキスト生成に必要
\usepackage{bxjalipsum}

\usepackage{cite}
\usepackage{multicol}
\usepackage{geometry}
\usepackage{caption}
% \usepackage{wrapfig}

\geometry{top=10mm,bottom=10mm,left=15mm,right=15mm}

\pagestyle{fancy}
\fancyhf{} % ヘッダー・フッターの初期化

% フッターの設定
\cfoot{\thepage} % ページ番号を中央に表示

\renewcommand{\headrulewidth}{0pt}

\begin{document}

\thispagestyle{fancy}

\noindent
\fbox{
  \begin{minipage}{\dimexpr\linewidth-2\fboxsep-2\fboxrule\relax}
    電気系修士輪講資料
    \hfill
    2025年 11月 14日
    % タイトル
    \begin{center}
      {\LARGE \bfseries スパイキングニューラルネットワークを用いた\\ドローン飛行制御に関する研究調査 \par}
      \vspace{2mm}
      {\large A Survey on Spiking Neural Network for Drone Flight Control \par}
    \end{center}

    \vspace{5mm}

    \noindent
    指導教員: 福田盛介 教授
    \hfill
    修士課程1年 37-256564 平田涼馬
  \end{minipage}
}

\vspace{5mm}

% アブストラクト
\begin{center}
  \bfseries Abstract
\end{center}
AI-based flight control for drones has gained significant attention for applications in unknown and dynamic environments. 
Particularly, in environments where communication delays are significant, such as Mars, there is a need for AI models that
can operate onboard, and the application of Spiking Neural Networks (SNNs) is anticipated. This survey analyzes recent research
related to SNN-based drone flight control, focusing on (1) methods using reinforcement learning, (2) methods using imitation learning,
and (3) methods combined with event-based vision sensors. Finally, we summarize and compare the key results of each approach and
discuss their future prospects to the flight control of Mars exploration drones.

\begin{multicols}{2}

% 本文
\section{序論}
ドローンは地形や障害物の影響を受けにくい高い移動能力や導入・運用コストの低さなどの特徴から農業や建設,
物流,惑星探査など幅広い用途への応用が進んでいる.これらの応用において,高い自律性を持つドローンの開発が
求められている.ドローンの飛行制御には,従来の制御理論に基づくPID制御やモデル予測制御などが用いられる
ことが一般的であったが,これらの手法には未知の環境や動的な環境への適応性,複雑なタスクの遂行能力に限界
がある.これらの課題に対処するため,近年ではニューラルネットワークを用いた制御手法に注目が集まっている
\cite{mcenroe2022survey}.ニューラルネットワークを用いたドローン制御ネットワークの推論は,通信を用いて
高性能なコンピューターにデータを送信して推論を行う方法と,ドローンに搭載したエッジデバイスで行う方法に大別
される.通信を用いる場合,高い計算能力を利用して高度な行動計画を立てたり,高い精度での物体認識などを活用
した制御が可能である一方で,通信遅延の影響からリアルタイム性に欠けるという課題がある.一方,エッジデバイス
を用いる場合,リアルタイムでの制御が可能であるが,電力の使用が飛行時間の短縮につながってしまう点や,
計算能力の不足からモデルの軽量化を行った結果,推論精度が低下するという課題がある\cite{rezwan2022artificial}.
これらの課題に対処するため,省電力かつ高効率な計算が可能なスパイキングニューラルネットワーク(SNN)を
用いたドローンの飛行制御手法が注目されている.

% \begin{center}
%   \includegraphics[width=\linewidth]{figures/SNN_operation.png}
%   \captionof{figure}{スパイキングニューラルネットワークの概念図 (\cite{SnnfromScratch}を一部改変)}
%   \label{fig:snn_operation}
% \end{center}

\subsection{スパイキングニューラルネットワーク}
スパイキングニューラルネットワーク(SNN)とは,生物の脳が{0,1}のスパイクを用いて情報伝達することを模倣した
AIモデルである.このスパイクを処理するニューロンのモデルとして使用されているのがLIFモデルである.
LIFモデルは,以下の式で表される.

\begin{equation}
  \tau_{m} \frac{dV(t)}{dt} = ( - V(t) + E_{rest}) + I(t)
  \label{equ:LIF_model}
\end{equation}

ここで,$V(t)$は膜電位,$E_{rest}$は静止膜電位,$I(t)$は入力電流,$\tau_{m}$は膜時定数を表す.
スパイクが入力されると,膜電位が上昇し,入力が無い場合は静止膜電位に向かって減衰する.連続でスパイクが
入力されることで,膜電位が閾値を超えるとニューロンが発火し,出力スパイクを生成する.SNNの特徴として,
入出力情報が{0,1}のスパイクで表現されることや,スパイクの入出力がある場合のみニューロンが活動するため,
専用のニューロモルフィックチップ上に実装した場合,計算量が少なく,省電力,リアルタイム処理が可能である
ことが挙げられる.一方で,SNNは入出力で扱うスパイクが微分不可能であるため,ANNで一般的に用いられる
誤差逆伝搬法を直接適用できないため,学習が困難であるという課題がある.現状,SNNの学習手法としては,
ANNで学習したパラメータをSNNに変換する手法や,代理勾配を用いて誤差逆伝搬法を適用する手法などが
用いられているが,ANNと比較して学習効率が低いという課題がある.また,ニューロモルフィックチップの性能も
発展途上であり,ANNと比較して計算能力が劣る場合が多い.

\subsection{火星探査への展望}
火星探査において,ドローンは地形の把握や科学観測,通信中継など多様な役割を担うことが期待されている.
NASAの火星ドローン「Ingenuity」は,火星大気中での飛行実証を成功させ,その後も複数の飛行ミッションを
遂行している\cite{Ingenuity}.将来的な火星探査ミッションでは,ドローンがより高度な自律飛行能力を
持つことが求められており,未知の地形や動的な環境に適応できる制御手法の開発が必要である.この制御手法
として,SNNを用いたドローン制御は,省電力かつリアルタイム処理が可能であるため,火星探査ドローンへの
応用が期待されている\cite{harbour2024martian}.

\begin{center}
  \includegraphics[width=\linewidth]{figures/Ingenuity_width.png}
  \captionof{figure}{火星ドローンIngenuity (\cite{Ingenuity}より引用)}
  \label{fig:ingenuity_on_mars}
\end{center}

\subsection{本調査の目的}
本調査では,SNNを用いたドローンの飛行制御について,シミュレーションによりSNNを用いた強化学習に取り組んだ
手法,模倣学習で訓練したSNNでセンサーデータをモーターコマンドにマッピングして制御を行う手法,入力に
イベントカメラと呼ばれるビジョンセンサーを利用した制御手法について紹介し,各手法の分析から将来の火星探査
ミッションにおけるドローン飛行制御への応用可能性について議論を行う.

\section{イベントベース入力によるSNN制御\cite{paredes2024fully}}
\subsection{概要}
この研究では,イベントカメラを用いて取得したイベントベースの視覚情報を入力とし,低レベルの制御アクションを
出力するSNNを提案している.

\subsection{}
本手法では,視覚処理を行うためのセンサーとしてイベントカメラを用いている.イベントカメラは,従来のフレーム
ベースのカメラとは異なり,ピクセル毎に独立して輝度変化を検出し,変化があった場合のみイベントとして出力する
センサーである.この特性から,$\mu s$オーダーの高い時間分解能,広いダイナミックレンジ,低消費電力などの
利点を持ち,スパイクで情報を処理するSNNと相性が良い.

\section{模倣学習SNNを用いたモーター制御\cite{stroobants2025neuromorphic}}
\subsection{概要}
この研究では,ドローンの自律制御を単一のニューロモルフィックチップのみで実現することを目的として,
IMUの入力からモーターコマンドを出力するSNNベースの制御システムを提案している.本手法では,学習時に
ネットワークを姿勢推定と制御の2つのサブネットワークに分割し,それぞれを教師あり学習,模倣学習で訓練
すること,SNNを用いることにより生じるセンサーバイアスやフィードフォワード遅延に対象するための工夫を
行っている.また,提案手法はCrazyflie上に実装され,評価が行われた.

\subsection{}
IMUは加速度センサーとジャイロスコープから構成され,ドローンの3自由度の加速度と角速度を提供する.
IMUからの入力は

\begin{center}
  \includegraphics[width=\linewidth]{figures/neuro_network.png}
  \captionof{figure}{提案手法のネットワーク構成 (\cite{stroobants2025neuromorphic})}
  \label{fig:neuro_network}
\end{center}

\subsection{}


\section{強化学習SNNを用いたナビゲーション\cite{lee2025bio}}

\section{まとめ}

\bibliographystyle{junsrt}
\bibliography{references}

\end{multicols}

\end{document}
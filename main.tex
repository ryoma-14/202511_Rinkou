%% utokyo eeis rinko template created by Ryoma Hirata

% documentclass
\documentclass[a4paper,10.5pt,dvipdfmx]{bxjsarticle}

\usepackage[dvipdfmx]{graphicx} % 画像の取扱いに必要
\usepackage{fancyhdr} % ヘッダー・フッターの設定に必要
\usepackage{lipsum} % ダミーテキスト生成に必要
\usepackage{bxjalipsum}

\usepackage{cite}
\usepackage{multicol}
\usepackage{geometry}

\geometry{top=10mm,bottom=10mm,left=15mm,right=15mm}

\pagestyle{fancy}
\fancyhf{} % ヘッダー・フッターの初期化

% フッターの設定
\cfoot{\thepage} % ページ番号を中央に表示

\renewcommand{\headrulewidth}{0pt}

\begin{document}

\thispagestyle{fancy}

\noindent
\fbox{
  \begin{minipage}{\dimexpr\linewidth-2\fboxsep-2\fboxrule\relax}
    電気系修士輪講資料
    \hfill
    2025年 11月 14日
    % タイトル
    \begin{center}
      {\LARGE \bfseries スパイキングニューラルネットワークを用いた\\ドローン飛行制御に関する研究調査 \par}
      \vspace{2mm}
      {\large A Survey on Spiking Neural Network for Drone Flight Control \par}
    \end{center}

    \vspace{5mm}

    \noindent
    指導教員: 福田盛介 教授
    \hfill
    修士課程1年 37-256564 平田涼馬
  \end{minipage}
}

\vspace{5mm}

% アブストラクト
\begin{center}
  \bfseries Abstract
\end{center}
AI-based flight control for drones has gained significant attention for applications in unknown and dynamic environments. 
Particularly, in environments where communication delays are significant, such as Mars, there is a need for AI models that
can operate onboard, and the application of Spiking Neural Networks (SNNs) is anticipated. This survey analyzes recent research
related to SNN-based drone flight control, focusing on (1) methods using reinforcement learning, (2) methods using imitation learning,
and (3) methods combined with event-based vision sensors. Finally, we summarize and compare the key results of each approach and
discuss their future prospects to the flight control of Mars exploration drones.

\begin{multicols}{2}

% 本文
\section{序論}
ドローンは地形や障害物の影響を受けにくい高い移動能力や導入・運用コストの低さなどの特徴から農業や建設,
物流,惑星探査など幅広い用途への応用が進んでいる.これらの応用において,高い自律性を持つドローンの開発が
求められている.ドローンの飛行制御には,従来の制御理論に基づくPID制御やモデル予測制御などが用いられる
ことが一般的であったが,これらの手法には未知の環境や動的な環境への適応性,複雑なタスクの遂行能力に限界
がある.これらの課題に対処するため,近年ではニューラルネットワークを用いた制御手法に注目が集まっている.


\subsection{スパイキングニューラルネットワーク}
スパイキングニューラルネットワーク(SNN)とは,生物の脳がスパイクを用いて情報伝達することを模倣した
AIモデルである.このスパイクを処理するニューロンのモデルとして最も使用されているのがLIFモデルである.
LIFモデルは,

\subsection{本調査の目的}
本紙では,SNNを用いたドローンの飛行制御について,シミュレーションによりSNNを用いた強化学習に取り組んだ
手法,模倣学習で訓練したSNNでセンサーデータをモーターコマンドにマッピングして制御を行う手法,入力に
イベントカメラと呼ばれるビジョンセンサーを利用した制御手法について紹介し,各手法の分析から将来の火星探査
ミッションにおけるドローン飛行制御への応用可能性について議論を行う.

\section{強化学習SNNを用いたナビゲーション}
あいう\cite{lee2025bio}

\section{模倣学習SNNを用いたモーター制御}
あいう\cite{stroobants2025neuromorphic}

\section{イベントベース入力によるSNN制御}
あいう\cite{paredes2024fully}
\subsection{ニューロモルフィック視覚処理}
\subsection{イベントカメラの概要}

\subsection{}
\section{まとめ}
あいう

\bibliographystyle{junsrt}
\bibliography{references}

\end{multicols}

\end{document}